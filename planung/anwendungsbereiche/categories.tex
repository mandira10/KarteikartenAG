\large{Kategorien UML Case Diagramm}
\begin{center}
    \begin{tikzpicture}
        \begin{umlsystem}[x=4, fill=red!10]{\systemname}
            \umlusecase[x=6, y=-1.5, width=2.5cm, name=edit]{Kategorie bearbeiten}
            \umlusecase[x=10, y=0, width=2.5cm, name=name]{Name setzen}
            \umlusecase[x=10, y=-3, width=2.5cm, name=poly]{Polyhierarchie definieren}
            \umlusecase[x=2, y=-3.5, name=create]{Kategorie anlegen}
			\umlusecase[x=4, y=-4.5, name=delete]{Kategorie löschen}
			\umlusecase[x=0, y=-10, name=overview]{Übersicht}
			\umlusecase[x=5, y=-9, name=sort]{Übersicht sortieren}
			\umlusecase[x=5, y=-11, name=view]{Polyhierarchie ansehen}
			\umlusecase[x=7, y=-10, name=select]{mehrere Kategorien auswählen}
			\umlusecase[x=10, y=-8, width=2.5cm,name=action]{Aktion über Kategorieauswahl}
			
			\umlusecase[x=6, y=-6.5, width=2.5cm,name=export]{Export als PDF/JSON/XML}
			\umlusecase[x=11, y=-6, width=2cm,name=deck]{Karteikasten erstellen}
        \end{umlsystem}

        % Actors
        \umlactor[y=-6]{User}
        
        % Associations
        \umlassoc[geometry=|-]{User}{edit}
        \umlinherit{name}{edit}
        \umlinherit{poly}{edit}
        \umlextend{create}{edit}
        \umlextend{delete}{edit}
        \umlassoc[geometry=|-]{User}{overview}
		\umlinclude{overview}{sort}
		\umlassoc{overview}{select}
		\umlassoc{overview}{view}
        \umlassoc{select}{action}
        \umlextend{action}{edit}
        \umlinclude{action}{export}
        \umlinclude{action}{deck}
    \end{tikzpicture}
\end{center}

