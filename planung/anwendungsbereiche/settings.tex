\large{Einstellungen UML Case Diagramm}
\begin{center}
    \begin{tikzpicture}
        \begin{umlsystem}[x=4, fill=red!10]{\systemname}
            \umlusecase[y=-10, name=setting-0]{Settings}
            \umlusecase[x=5, y=-3.5, width=1.5cm, name=setting-1]{Sprache wählen}
            \umlusecase[x=10, y=-20, width=1.5cm, name=setting-2]{Check if already logged in}
            \umlusecase[x=5, y=-6, width=1.5cm, name=setting-3]{Exportiert Kategorie als PDF}
            \umlusecase[x=10, y=-5, width=1.5cm, name=setting-4]{Deutsch}
            \umlusecase[x=10, y=-3, width=1.5cm, name=setting-5]{Englisch}
            \umlusecase[x=5, y=-8.7, width=1.5cm, name=setting-6]{Exportiert Kategorie als JSON}
            \umlusecase[x=5, y=-11.5, width=1.5cm, name=setting-7]{Exportiert Kategorie als XML}
            \umlusecase[x=4, y=-18, width=1.5cm, name=setting-8]{Logout}
            \umlusecase[x=10, y=-7.5, width=1.5cm, name=setting-10]{Die aktuelle Kategorie}
            \umlusecase[x=10, y=-11, width=1.5cm, name=setting-11]{Eine andere Kategorie}
            \umlusecase[x=5, y=-14, width=1.5cm, name=setting-12]{Hellmodus}
            \umlusecase[x=5, y=-16, width=1.7cm, name=setting-13]{Dunkelmodus}
            \umlusecase[x=10, y=-14, width=1.7cm, name=setting-14]{Waehle eine Kategorie}
            
        \end{umlsystem}

        % Actors
        \umlactor[y=-6]{User}
        \umlactor[x=17, y=-6]{\servername}


        % Settings
        \umlassoc[geometry=|-]{\clientname}{setting-0}
        \umlextend{setting-1}{setting-0}
        
        \umlextend{setting-8}{setting-0}
        \umlVHinclude{setting-2}{setting-8}
        
        \umlextend{setting-4}{setting-1}
        \umlextend{setting-5}{setting-1}
        
        \umlassoc[geometry=|-]{\servername}{setting-8}
        
        \umlextend{setting-6}{setting-0}
        \umlextend{setting-7}{setting-0}
        \umlextend{setting-3}{setting-0}
        
        
        \umlextend{setting-12}{setting-0}
        \umlextend{setting-13}{setting-0}
        
        \umlextend{setting-10}{setting-3}
        \umlextend{setting-10}{setting-6}
        \umlextend{setting-10}{setting-7}
        
        \umlextend{setting-11}{setting-3}
        \umlextend{setting-11}{setting-6}
        \umlextend{setting-11}{setting-7}
        
        \umlinclude{setting-11}{setting-14}
        
    \end{tikzpicture}
\end{center}

\begin{answer}
    Detailanwendungsfall die Sprache auf Englisch umstellen 
\end{answer}
\newline
\textbf{Akteure}: Schüler Sam
\newline
\textbf{Vorbedingungen}: 
\begin{itemize}  
    \item Das System ist gestartet.
    \item Sam hat sich nicht gemeldet.
   \item Das System ist auf Deutsch.
\end{itemize} 
\textbf{Nachbedingungen}: 
\begin{itemize}  
    \item Er hat die App Sprache erfolgreich von Deutsch auf Englisch geändert.
   \item Oder, er kann nicht finden, wo man die Sprache ändern kann.\\
\end{itemize}
\textbf{Regulärer Ablauf}: 
\begin{itemize}  
    \item Sam klickt auf den Button Settings.
   \item Weitere Funktionen von Settings werden angezeigt. (siehe Abb.)
   \item Sam bringt sein Mauszeiger zu Sprachen.
    \item Weitere Funktionen von Sprachen werden angezeigt.
    \item Sam klickt auf den Button Englisch.
     \item Die Sprache wurde auf Englisch umgestellt.
\end{itemize}
\textbf{Alternative}: 
\begin{itemize}  
    \item Er hat nicht genug Deutschkentnisse, das Wort ,,Sprachen'' zu verstehen. Deswegen bringt er sein Mauszeiger nicht zu Sprachen und kann das Button ,,Englisch" nicht finden.
\end{itemize} 