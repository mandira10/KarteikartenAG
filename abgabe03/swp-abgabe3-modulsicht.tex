% !TeX root = swp-abgabe3-modulsicht.tex
\documentclass[fontsize=12pt,paper=A4,twoside]{scrartcl}

% SWP-Präambel
% C 2003-2021 Sebastian Offermann, Rainer Koschke, Karsten Hölscher
% In Zeilen 41 und 42 sind jeweils die aktuellen Semester-Daten einzutragen.

\usepackage[utf8]{inputenc}     % Kodierung der Tex-Datei
\usepackage[T1]{fontenc}        % Korrekte Ausgabe von Sonderzeichen (Umlaute)
\usepackage[ngerman]{babel}     % Deutsche Einstellungen [ab \begin{document}]

\usepackage{bibgerm}            % Bibliographie
\usepackage{fancyhdr}           % obere Seitenränder gestalten
\usepackage{float}              % Floats Objekte mit [H] festsetzen
\usepackage{graphicx}           % Graphiken als jpg, png etc. einbinden
\usepackage{moreverb}           % zusätzliche verbatim-Umgebungen
\usepackage{pdflscape}          % PDF-Support für landscape
\usepackage[final]{pdfpages}    % Externe PDFs einbinden
\usepackage{stmaryrd}           % zusätzliche Symbole
\usepackage{supertabular}       % Tabellen über Seitenränder hinaus
\usepackage{tabularx}           % Tabellen mit vorgegebener Breite
\usepackage{url}                % setzt URLs schön mit \url{http://bla.laber.com/~mypage}

%%% Die Reihenfolge der folgenden Pakete muss beibehalten werden:
%%% varioref, hyperref, cleveref, bookmark
% Verweise innerhalb des Dokuments schick mit " ... auf Seite ... "
% automatisch versehen. Dazu \vref{labelname} benutzen
\usepackage[ngerman]{varioref}  % [vor hyperref für korrekte Verweise]
\usepackage[colorlinks=true, pdfstartview=FitV, linkcolor=blue,
            citecolor=blue, urlcolor=blue, hyperfigures=true,
            pdftex=true]{hyperref} % [vor bookmark wegen der Optionen]
\usepackage[ngerman]{cleveref}
\usepackage{bookmark}

\hyphenation{Arbeits-paket}     % Trennungsregeln

%%% Definitionen
\newcommand{\grad}{\ensuremath{^{\circ}} }
\renewcommand{\strut}{\vrule width 0pt height5mm depth2mm}
\newcommand{\gq}[1]{\glqq{}#1\grqq{}}
\newcommand{\skipInput}[1]{}

%%% Semesterkonstanten
\newcommand{\sem}{WiSe}%{SoSe}
\newcommand{\jahr}{2022/23} %2017/2018

\newif\iftoc
\tocfalse % für kleine Abgaben kein Inhaltsverzeichnis nötig
%\toctrue % für größere Abgaben mit Inhaltsverzeichnis

\newcommand{\highlight}[1]{\textcolor{blue}{\textbf{#1}}}
\newcommand{\swp}{Software-Projekt}
\newcommand{\semester}{\sem\ \jahr}

%%% Formatierungsanpassungen
% Damit Latex nicht zu lange Zeilen produziert:
\sloppy
%Uneinheitlicher unterer Seitenrand:
%\raggedbottom

% Kein Erstzeileneinzug beim Absatzanfang
% Sieht aber nur gut aus, wenn man zwischen Absätzen viel Platz einbaut
\setlength{\parindent}{0ex}

% Abstand zwischen zwei Absätzen
\setlength{\parskip}{1ex}

% Seitenränder für Korrekturen verändern
\addtolength{\evensidemargin}{-1cm}
\addtolength{\oddsidemargin}{1cm}

\bibliographystyle{gerapali}

\newcommand\documentTitle{Bitte documentTitle festlegen!}
\newcommand\groupName{Bitte groupName festlegen!}
% swpdocument verwendet die Werte von documentTitle und groupName,
% entsprechend sollten diese vorher umgesetzt werden; sonst wird eine
% Erinnerungsmeldung an der entsprechenden Stelle im Dokument platziert
% 1. Parameter: Euer/Eure Tutor:in, z. B. {Kim Harrison}
% 2. Parameter: Abgabedatum, z. B. {05. April 2063}
% 3. Parameter: Versionsnummer, z. B. {1.1}
% 4.-9. Parameter: jeweils Name und (Uni-)Email-Adresse jedes 
%                 Gruppenmitglieds; mit einem & getrennt, z. B.
% {Robin Cowl & roco@uni-bremen.de}
% Besteht die Gruppe aus weniger als 6 Personen, so werden die 
% übrigen Parameter leer gelassen: {}
\newcommand \swpdocument[8] {
% Lustige Header auf den Seiten
  \pagestyle{fancy}
  \setlength{\headheight}{70.55003pt}
  \fancyhead{}
  \fancyhead[LO,RE]{\swp{}\\%
                    \semester{}\\%
                    \documentTitle}
  \fancyhead[LE,RO]{Seite \thepage\\%
                    \slshape \leftmark\\%
                    \slshape \rightmark}

% Lustige Header nur auf dieser Seite (Titelseite)
  \thispagestyle{fancy}
  \fancyhead[LO,RE]{ }
  \fancyhead[LE,RO]{Universität Bremen\\%
                    FB 3 -- Informatik\\%
                    Tutor: #1}
  \fancyfoot[C]{}

% Start Titelseite
  \vspace{3cm}
  \begin{minipage}[H]{\textwidth}
    \begin{center}
      \bfseries \Large \swp{} -- \semester{}\\
      \smallskip
      \small 03-IBGP-SWP\\
      \vspace{3cm}
    \end{center}
  \end{minipage}
  \begin{minipage}[H]{\textwidth}
    \begin{center}
      \vspace{1cm}
      \bfseries {\Large \documentTitle}\\
      \vspace{3ex}
      $<$\groupName$>$\\%
      \vfill
    \end{center}
  \end{minipage}
  \vfill
  \begin{minipage}[H]{\textwidth}
    \begin{center}
      \sffamily
      \begin{tabular}{lr}
        #3 \\
        #4 \\
        #5 \\
        #6 \\
        #7 \\
        #8 \\
      \end{tabular}
      \\[22mm]
      \itshape Abgabe: #2\\ ~
    \end{center}
  \end{minipage}
% Ende Titelseite

\iftoc
%\iffalse
% Start Inhaltsverzeichnis
\newpage
  \thispagestyle{fancy}
  \fancyhead{}
  \fancyhead[LO,RE]{\swp{}\\%
                    \semester{}\\%
                    \documentTitle}
  \fancyhead[LE,RO]{Seite \thepage\\%
                    \slshape \leftmark\\~}
  \fancyfoot{}
  \renewcommand{\headrulewidth}{0.4pt}
\tableofcontents
% Ende Inhaltsverzeichnis
\fi

% Header für alle weiteren Seiten
\newpage
  \fancyhead[LE,RO]{Seite \thepage\\%
                    \slshape \leftmark\\%
                    \slshape \rightmark}

}
\usepackage[shortlabels]{enumitem}
\usepackage{../tikz-uml}
\usetikzlibrary{shapes.misc}
%\graphicspath{{./images/}}


% Document start
\begin{document}
    \renewcommand\documentTitle{Modulsicht}
    \renewcommand\groupName{KarteikartenAG}

    \newcommand\cat[1]{\textbf{\large #1}\\[0.5em]}
    \newcommand\cata[1]{\textbf{#1}\\[0.3em]}

    \swpdocument{Rodrigue Wete Nguempnang}{04. Dezember 2022}%
                {Mert As & meras@uni-bremen.de}%
                {Tom Beuke & tombeuke@uni-bremen.de}%
                {Efe Carkcioglu & efe1@uni-bremen.de}%
                {Nadja Cordes & ncordes@uni-bremen.de}%
                {Ole-Niklas Mahlstädt & olma@uni-bremen.de}%
                {Henry Zöllner & henry5@uni-bremen.de}%

    \sffamily
    \KOMAoptions{paper=A4,paper=portrait,pagesize}
    \section{Modulsicht Beschreibung}\label{sec:detailliert:Modulsicht}

    \cat{Module}
    Unsere Module sind in fünf unterschiedliche Module mit teils Untermodulen aufgeteilt.
    \begin{itemize}
        \item \textbf{GUI} mit den Untermodulen: CardGUI, DeckGUI und CategoryGUI mit jeweils mehreren Klassen.
        \item \textbf{Controller} mit den Klassen: CardController, DeckController, CategoryController.
        \item \textbf{Logic} mit den Klassen. CardLogic, DeckLogic, CategoryLogic.
        \item \textbf{Persistence} mit den Klassen CardRepository, DeckRepository, CategoryRepository.
        \item \textbf{Datenmodell}
    \end{itemize}


    \cat{GUI}
    Die \texttt{GUI} ist zuständig für die Darstellung unseres Karteikartensystems.
    Sie ist aufteilt in mehrere Untermodule für die einzelnen Hauptbestandteile
    unserer Software: Die Karteikarten, die Decks zum Lernen der Karteikarten und die Kategorien,
    denen die Karten zugeordnet werden. Diese Module enthalten die einzelnen Klassen zur Darstellung
    unseres Systems.\\\\
    \cata{CardGUI} \texttt{CardGUI} beinhaltet 4 Klassen. In der \texttt{CardOverviewPage} werden
    alle Karten dargestellt (Glossar). Wenn man neue Karten anlegen oder bestehende bearbeiten will, tut
    man dies über die \texttt{EditCardPage}, die über das Glossar geöffnet werden kann. Auch eine Einzelkartensicht
    kann über das Glossar aufgerufen, der User wird dann zur \texttt{ViewSingleCardPage} weitergeleitet.
    Die letzte Seite wird beim Lernmodus genutzt die \texttt{TestCardPage}.\\\\
    \cata{DeckGUI} \texttt{DeckGUI} beinhaltet 3 Klassen. Die Standardseite ist \texttt{DeckOverviewPage},
    in der alle zur Verfügung stehenden Decks aufgelistet werden. Über diese kann auf die \texttt{EditDeckPage}
    navigiert werden, worüber neue Decks erstellt werden oder aber bestehende angepasst werden. Auch kann sich ein
    User ein einzelnes Deck über \texttt{ViewSingleDeckPage} anzeigen lassen mitsamt der wichtigsten Informationen
    zum Deck.\\\\
    \cata{CategoryGUI} \texttt{CategoryGUI} beinhaltet 4 Klassen. \texttt{CategoryOverviewPage} listet alle Kategorien
    und wie oft sie verwendet werden auf. Über die Klasse \texttt{ViewCategoryHierarchyPage} kann
    der Nutzer die gesamte Hierachie der Klassen mit \texttt{Parents} und \texttt{Children} einsehen.
    Auch hier gibt es analog zu \texttt{Deck} und \texttt{Card} eine extra Seite für das Erstellen und Bearbeiten
    (\texttt{EditCategoryPage}) sowie eine Einzelkategorieübersicht (\texttt{ViewSingeCategoryPage}).

    \newpage
    \cat{Controller}
    Der \texttt{Controller} steht im Austausch mit der \texttt{GUI} und gibt Aufrufe, die vom Nutzer
    in der \texttt{GUI} gemacht werden, an die \texttt{Logik} weiter.\\
    Er teilt sind in 3 Klassen: \texttt{CardController}, \texttt{DeckController} und \texttt{CategoryController}, die alle Aufrufe, die von den 3 Gui-Untermodulen kommen, bearbeiten.

    \cat{Logik}
    Die \texttt{Logik} ist das Modul, dass sich um die Ausführung von den Anwendungsfällen kümmert.
    Sie leitet Datenbezogene Aufgaben an die \texttt{Persistence} weiter.\\\\
    Die Logik teilt sind in 3 Klassen: \texttt{CardLogic}, \texttt{DeckLogic} und \texttt{CategoryLogic}. Diese übernehmen
    die Ausführung für die einzelnen Submodule.

    \cat{Persistence}
    Die \texttt{Persistence} ist das Modul, dass datenbezogene Arbeiten ausführt. Das Laden und Speichern wird
    an unseren Server weitergegeben.\\\\
    Die Persistence teilt sind in 3 Klassen: \texttt{CardRepository}, \texttt{DeckRepository} und \texttt{CategoryRepository}, die das Laden und Speichern der
    Submodule ausführen, indem sie mit der Datenbank auf dem Server kommunizieren.



    \clearpage
    \section{Modulsicht Detailbeschreibung}
    Da die Methoden aufeinander aufbauen, wird hier jeweils die gedachte Funktionsweise von GUI bis Persistence beschrieben.\\
    Die spätere Funktion der Datenbank, die bei uns über den Server läuft, ist hier noch nicht inkludiert.\\
    Auch gibt es Zusatzfunktionen, die nicht vollends dargestellt sind, wie bspw. die Exportfunktion, die beschrieben wird,
    aber der Übersicht halber nicht in der Modulsicht erscheint. \\

    \textbf{\large Karte}
    \begin{itemize}
    \item \textit{N1 Karteikarte anlegen/\\ N3 Karteikarte bearbeiten}\\ 
    Siehe Beschreibung in Sequenzdiagramm Teil 1. Nach Aufruf wird ein \texttt{boolean} zurückgegeben, ob erfolgreich.
    \item \textit{N2 Karteikarte löschen/\\ N23 Kartenauswahl löschen}\\ 
    Kann entweder bei Kartenbearbeitung direkt (\texttt{CardEditPage} - siehe Sequenzdiagramm Teil 1) oder über das Glossar (\texttt{CardOverviewPage}) 
    für einzelne oder mehrere Karten über den \texttt{CardController} aufgerufen werden. Nach Aufruf wird ein \texttt{boolean} zurückgegeben, ob erfolgreich.
    \item \textit{N4 Karteikarte als PDF exportieren/\\ 
    N7 Karteikarte als JSON exportieren /\\
    Karteikarte als XML exportieren /\\
    N24 Kartenauswahl exportieren}\\ 
    Für das Exportieren gibt es eine separate Seite \texttt{ExportSettingsPage}, die nach Aufruf 
    in verschiedenen Übersichten aufgerufen und über die \texttt{CardLogic} verarbeitet wird. In der \texttt{Persistence} Klasse gibt es einen \texttt{Exporter}, der die Befehle 
    verarbeitet und je nach Art des Exports einen der drei spezifischen Exporter in \texttt{Persistence} aufruft, die den eigentlichen Export behandeln.
    \item \textit{N19 Karteikarten anzeigen}\\ Für das Anzeigen der Karteikarten benötigt man zwei Methoden. Initial werden die
    Karten über \texttt{getCardsToShowInitially()} geladen, die im Controller verarbeitet wird und an die Logik weitergereicht werden. Für die gezogenen Karten wird
    dann über den \texttt{DeckController} noch die Anzahl der Karten abgefragt, was ebenfalls über die Logik erfolgt.
    \item \textit{N20 Karteikartenansicht verändern\\- Filterung}\\ 
    Für die Filterung der Karte gibt es drei Methoden: Bei Filterung nach Tag wird im \texttt{CardController} \texttt{getCardsByTag(tag)} aufgerufem.
    Für die Suchbegriffe wird im \texttt{CardController} \texttt{getCardsBySearchTerms(String)} aufgerufen.die \texttt{CardLogic} übergebenen Methoden werden an den Server über die \texttt{CardRepository} weiterrgeben. \texttt{SearchWords} werden über 
    den Server mit den \texttt{getContent} Methoden der \texttt{Card} verglichen.
    Die Filterung nach \texttt{Category} wird über den \texttt{CategoryController} behandelt und dort weitergereicht.
    \end{itemize}
    \ 

    \textbf{\large Deck}
    \begin{itemize}
    \item\textit{N25 Karteikasten einzelne Kategorie / \\
    N26 Karteikasten mehrere Kategorien erstellen\\
    N27 Karteikasten bearbeiten\\
    N28 Lernsystem auswählen}\\
    Analog wie bei \texttt{Card} gibt es eine gebündelte Methode für die Bearbeitung und Erstellung von Karteikästen.
    Sollte der Nutzer später sein Lernsystem ändern, wird dies auch über die Methode aufgerufen in der \texttt{GUI} aber später in der \texttt{DeckLogic}
    separat behandelt über \texttt{updateStudySystem}.
    Zuvor wird, wie bei \texttt{Card} auch, die unterschiedlichen Lernsysteme abgerufen, damit sie ausgewählt werden können.
    \item \textit{N31 Kasten Lernen/ \\
    N32 Lernvorgang unterbrechen/ \\
    N33 Lernvorgang fortsetzen/ \\
    N34 Lernfortschritt/ \\
    N35 Antwort prüfen} \\ Siehe Beschreibung in Sequenzdiagramm Teil 2.
    \item \textit{N28-4 Neue Lernmethoden erstellen}\\
    Für die Funktion neue Lernmethoden anzulegen, gibt es eine extra Seite, die über die \texttt{DeckOverviewPage} aufgerufen werden kann: \texttt{EditStudySystemTypePage}.
    Über diese können auch bestehende custom Systeme angepasst werden und neue erstellt werden. Nach Anpassung der Kartenboxen sowie der Wiederholungsintervalle wird über \texttt{applyChanges} 
    wieder an den Controller weitergereicht und in der \texttt{DeckLogik} werden dann zwei Methoden ausgeführt: Zum einen wird das neue StudySystem hinzugefügt und geprüft, ob es neu ist, zum anderen
    werden alle StudySystems geupdated auf dem Cache.
    \item \textit{N35 Status der Karteikarten/ \\ N36 Karteikastenübersicht}\\ 
    Für die Karteikästen gibt es, wie bei Karteikarten auch, eine Übersicht, wo zu jedem Karteikasten seine Karten und ihr Status angezeigt werden.
    Dies wird über die Methode \texttt{showDecks} in \texttt{DeckOverviewPage} realisiert. Hier wird die Methode \texttt{getCardToDecks} im \texttt{DeckController}, 
    die in der \texttt{DeckLogic} bearbeitet wird und weiter aufgesplittet. Zunächst wird dort \texttt{getDecksAndCards} aufgerufen, in der dann Zunächst
    die Decks mittel \texttt{getDecks} alle Decks gezogen werden und dann für jedes Deck mit \texttt{getCardsToDeck} die einzelnen Karten mit Status
    ermittelt werden.
    \item\textit{ N38 Karteikasten löschen}\\ Wird über \texttt{DeckOverviewPage} oder einzeln über \texttt{EditDeckPage} aufgerufen, für einen oder mehrere Karteikästen (analog zu \texttt{Card})
    \end{itemize}
    \newpage
    \textbf{\large Category}
    \begin{itemize}
        \item \textit{N12 Kategorie anlegen/ \\
         N14 Kategorie bearbeiten}\\
    Analog wie bei Karteikarten und Karteikästen gibt eine Methode für das Bearbeiten und Erstellen von Kategorien.
        \item \textit{N14 Kategorie löschen}\\
    Eine Kategorie kann über \texttt{EditCategoryPage} gelöscht werden. Verfahren analog wie bei \texttt{Card}.
    \end{itemize}

    \clearpage
    \section{Sequenzdiagramm Detailbeschreibung}
    \cat{Sequenzdiagramm 1: Karte erstellen}
    Susi möchte eine Karte erstellen und ruft über die Kartenübersicht Karte erstellen auf.
    \begin{enumerate}
    \item Da die \texttt{EditCardPage} sowohl zur Bearbeitung als auch zur Erstellung von Karten benutzt wird, prüft der \texttt{CardController}
    zunächst, ob eine Karte mitgegeben wurde und lädt die Informationen, wenn diese nicht leer ist. Dies wird an die \texttt{CardLogic} weitergegeben, die 
    den Befehl ans \texttt{CardRepository} leitet, dass die Funktion prüft, indem es auf dem Server auf die Datenbank zugreift.
    \item Wenn es eine Bearbeitung einer bestehenden Karte ist, wird nach Rückgabe ein Löschbutton für die Karte angezeigt.
    \item Da die bereits bestehenden Kategorien und Tags zur Auswahl im Dropdown angegeben werden sollen, werden auch diese abgerufen.
    Hierbei wird für die Kategorien der \texttt{CategoryController} verwendet.
    \item Nach Laden aller notwendigen Informationen kann die Bearbeitung erfolgen, hierbei kann jederzeit abgebrochen werden.
    \item Wenn Susi auf Übernehmen klickt, wird im \texttt{CardController} \texttt{updateCardData()} aufgerufen. Auch hier folgt
    in der Logik wieder eine Prüfung, ob es sich um eine bestehende oder eine neue Karte handelt und je nachdem wird im \texttt{CardRepository}
    \texttt{saveCard(Card)} oder \texttt{updateCard(Card)}.
    \end{enumerate}

    \newpage
    \cat{Sequenzdiagramm 2: Lernsystem lernen}
    Susi möchte einen bereits erstellen Karteikasten testen.
    \begin{enumerate}
    \item Dazu gelangt sie zunächst zur \texttt{TestDeckPage}. Über den Controller wird \texttt{getDeck()} aufgerufen und das Deck über die Persistence und den Server geladen.
    \item Über das StudySystem wird die nächste zu bearbeitende Karte geladen. Falls bereits vorher gelernt wurde, beginnt man beim vorherigen Lernfortschritt.
    \item Sobald das Laden fertig ist, startet der Test
    \item Hierbei gibt es zwei Optionen:
        \begin {enumerate}
        \item  Wenn es sich z.B. um eine Wahr/Falsch Frage handelt, weiß das System, welche Antwort es gibt. 
        \item Bei anderen Typen, wie bspw. der Text Frage, weiß das System es nicht. 
        \end {enumerate}
     \item Daher gibt es eine optionale Selbstbewertung, wo Susi die Antwort aufdecken muss und die \texttt{EditTestPage} die Antwort anzeigt.
     \item Danach wird die Antwort bestätigt von Susi, bzw. abgeschickt bei einer Frage mit bekanntem Ergebnis und ein \texttt{onAnswer} zurückgegeben, ob die Antwort falsch 
    oder richtig war.
    \item Abhängig von der Antwort wird die Karte dann über \texttt{mmoveCardToBox} in \texttt{StudySystem} intern verschoben.
    \item Darauf gibt es wieder zwei Möglichkeiten: 
        \begin {enumerate}
        \item Die Karte war die letzte im Stapel. Das System wird gefinished und der Lernfortschritt über den \texttt{DeckController} gespeichert. Dieser leitet 
        \texttt{updateStudySystem} weiter an die \texttt{DeckLogic}, die die Funktion an den Server weitergibt, der es dort in der Datenbank speichert.
        \item Es gibt eine weitere Karte zu lernen. Diese wird angezeigt und das Lernen wird fortgesetzt wie oben bereits beschrieben.   
        \end {enumerate}
    \item Zudem kann Susi jederzeit das Lernen abbrechen. Wenn Sie dies auswählt, dann muss der Lernfortschritt vom StudySystem ebenso auf dem Server gespeichert werden. Dafür wird 
     \texttt{updateStudySystem} aufgerufen, wie bei \texttt{finish}.
    \end{enumerate}

    



   
     \newpage
    \recalctypearea

    %Only to show how the structure is
    %some methods are still missing, folder implementation has more detail
    %\KOMAoptions{paper=A3,paper=landscape,pagesize}
    %\thispagestyle{empty}
    %\begin{tikzpicture}%
    %    \definecolor{beaublue}{rgb}{0.74, 0.83, 0.9}
\begin{umlpackage}[fill=beaublue]{GUI}
    \begin{umlpackage}[fill=beaublue]{CardGUI}
        \umlemptyclass[x=0, y=0]{CardOverviewPage}
        \umlemptyclass[x=0, y=-2]{EditCardPage}
        \umlemptyclass[x=0, y=-4]{ViewCardPage}
        \umlemptyclass[x=0, y=-6]{TestCardPage}
    \end{umlpackage}{CardGUI}
    \begin{umlpackage}[fill=beaublue]{DeckGUI}
       \umlemptyclass[x=5, y=0]{DeckOverviewPage}
       \umlemptyclass[x=5, y=-2]{EditDeckPage}
       \umlemptyclass[x=5, y=-4]{ViewSingleDeckPage}
    \end{umlpackage}{DeckGUI}   
    \begin{umlpackage}[fill=beaublue]{CategoryGUI}
        \umlemptyclass[x=11, y=0]{CategoryOverviewPage}
        \umlemptyclass[x=11, y=-2]{EditCategoryPage}
        \umlemptyclass[x=11, y=-4]{ViewCategoryHierarchyPage}
        \umlemptyclass[x=11, y=-6]{ViewSingeCategoryPage}
     \end{umlpackage}{CategoryGUI}   
\end{umlpackage}{GUI}

\begin{umlpackage}[fill=beaublue]{Controller}
    \umlemptyclass[x=0, y=-10]{CardController}
    \umlemptyclass[x=5, y=-10]{DeckController}
    \umlemptyclass[x=11, y=-10]{CategoryController}
\end{umlpackage}{Controller}

\begin{umlpackage}[fill=beaublue]{Logic}
    \umlemptyclass[x=-0.5, y=-14]{CardLogic}
    \umlemptyclass[x=5, y=-14]{DeckLogic}
    \umlemptyclass[x=11, y=-14]{CategoryLogic}
\end{umlpackage}{Logic}

\begin{umlpackage}[fill=beaublue]{Persistence}
    \umlemptyclass[x=0, y=-18]{CardRepository}
    \umlemptyclass[x=5, y=-18]{DeckRepository}
    \umlemptyclass[x=11, y=-18]{CategoryRepository}
\end{umlpackage}{Persistence}

\begin{umlpackage}[x=20, y=0, fill=beaublue]{Datamodel}
\end{umlpackage}{Datamodel}
    %\end{tikzpicture}
    \newpage
    %Only to show how the structure is
    %some methods are still missing, folder implementation has more detail
    \KOMAoptions{paper=A2,paper=landscape,pagesize}
    \thispagestyle{empty}
    \begin{tikzpicture}%
        \tikzset{cross/.style={cross out, draw, 
         minimum size=2*(#1-\pgflinewidth), 
         inner sep=0pt, outer sep=0pt}}

        %\definecolor{columbiablue}{rgb}{0.61, 0.87, 1.0}
\definecolor{beaublue}{rgb}{0.74, 0.83, 0.9}
\begin{umlpackage}[fill=beaublue, x=-0.5, y=0]{GUI}
    \begin{umlpackage}[fill=beaublue]{CardGUI}
        \umlclass[x=-3, y=0]{CardOverviewPage}
{ 
    + showCardsWithCategory(): void \\ 
    + showCardsWithTag(): void\\
    + showCardsWithSearchWords(): void \\
    + getCountOfDecksToCard():void\\
    + getCardsToShowInitially(): void\\
}{}
        \umlclass[x=5, y=2]{EditCardPage}
{ 
    - deleteCard(): void\\
    - applyChanges(): void \\
}{}
        \umlclass[x=-2, y=-4]{ViewCardPage}
{ 
    + displayInformationToCard(): void\\
}{}
        \umlclass[x=5, y=-3]{TestCardPage}
{
    + displayInformationToCard(): void\\
}{}
    \end{umlpackage}{CardGUI}
    \begin{umlpackage}[fill=beaublue, x=0, y=0.75]{DeckGUI}
        \umlclass[x=12, y=0]{DeckOverviewPage}
{ 

    - exportCards(): void \\
}{}
        \umlclass[x=19, y=0]{EditDeckPage}
{ 
    + editDeck(): void \\ 
    + deleteDeck(): void\\
    + addCardToDeck(): void \\
    + addCardsOfCategoryToDeck(): void \\
}{}
        \umlclass[x=17, y=-2]{StudySystemPage}
{ 
    - deleteStudySystem(): void \\
    - applyChanges(): void \\
}{}
        \umlclass[x=17, y=-5]{TestDeckPage}
{ 
    + startTests(Deck): void \\
    - nextCard(): void \\
    - finishTest(): void \\
    + update(): void \\
}{}
        \umlclass[x=12, y=-3]{ViewSingleDeckPage}
{ 
    + displayInformationToDeck: void \\ 
}{}
    \end{umlpackage}{DeckGUI}
	\begin{umlpackage}[fill=beaublue]{CategoryGUI}
		\umlclass[x=27]{CategoryOverviewPage}{
	+ showCategories(): void \\
	+ getCountOfCardToCategory(): void \\
	+ exportCards(): void
} {}
		\umlclass[x=34]{EditCategoryPage}{
	+ createCategory(): void \\
	+ editCategory(): void \\
	+ deleteCategory(): void \\
	+ addCardToCategory(): void
}{}
		\umlclass[x=27,y=-3]{ViewCategoryHierarchyPage}{
	+ viewCategoryHierarchyPage)(): void \\
	+ updateTopLevelCategories(): void
}{}
		\umlclass[x=34,y=-3]{ViewSingleCategoryPage}{
	+ displayInformationForCard(): void
}{}
	\end{umlpackage}
\end{umlpackage}{GUI}

\begin{umlpackage}[fill=beaublue, x=3.75, y=0]{Controller}
    \umlclass[x=-2, y=-12]{CardController}
{
    + getCardsByTag(Tag): Set<Card> \\
    + getCardsBySearchterms(String): Set<Card> \\
    + getCardByUUID(String): Card \\
    + deleteCard(Card): boolean \\
    + deleteCards(Card[]): boolean \\
    + createTag(String): boolean \\
    + getTags(): Set<Tag> \\
    + updateCardData(Card, Card): boolean \\
    + createCardToTag(Card, Tag): boolean \\
    + getCardsToShow(long, long): Set<Card> \\
    + exportCards(Card[], ExportFileType): boolean \\
}{}
    \umlclass[x=12, y=-8.5]{DeckController}
{ 
    + updateDeckData(Deck, Deck): void \\ 
    + deleteDeck(Deck): void \\ 
    + deleteDecks(Deck[]) \\ 
    + getCards() : Card[]\\
    + createCardToDeck(Card, Deck): void \\
    + createCardToDeckForCategory(Category, Deck): List<Deck>\\
    + getAllInfosForDeck(String): void\\
    + getCountOfCardsFor(Deck): void\\
    + exportCards(Card[], ExportFileType): void
}{}
    \umlclass[x=28,y=-12]{CategoryController}{
	+ updateCategoryData(Category, Category): boolean \\
	+ createCardToCategory(Card, Category): boolean \\
	+ deleteCategory(Category): boolean \\
	+ deleteCardToCategory(CardToCategory) \\
	+ getCategories(): Set<Category> \\
	+ getCategoryByUUID(String): Category \\
	+ getCardsInCategory(Category): Set<Card> \\
	+ numCardsInCategory(Category): int \\
}{}
\end{umlpackage}{Controller}0.5

\begin{umlpackage}[fill=beaublue, x=5.25, y=0.75]{Logic}
    \umlclass[x=-1, y=-21.5]{CardLogic}
{ 
    + getCardsByTag(Tag): Set<Card> \\
    + getCardByUUID(String): Card \\
    + getCardsBySearchterms(String): Set<Card> \\
    + getCardsToShow(long, long): Set<Card> \\
    + createCardToTag(Card, Tag): boolean \\
    + createTag(String): boolean \\
    + getTags(): Set<Tag> \\
    % -changeCard(): void
    + updateCardData(Card, Card): boolean \\
    + deleteCard(Card): boolean \\
    + deleteCards(Card[]): boolean \\
    + exportCards(Card[], ExportFileType): boolean \\
}{}
    \umlclass[x=8.75, y=-20]{DeckLogic}
{ 
<<<<<<< HEAD
    + getDecks(): List<Deck> \\ 
    + updateDeckData(Deck, Deck): void\\
    + updateStudySystem(Deck, StudySystem): void\\
    + deleteDeck(Deck):void\\
    + deleteDecks(Deck[]):void\\
    + getCountOfDecksFor(String): int\\
    + getAllInfosForDeck(String):Deck\\
    + getCountOfCardsInDeck(Deck):Deck\\
    + getCardsToDecks(): List<CardToDeck>\\
}{} 
=======
    + getDecks(): Set<Deck> \\
    + createCardToDeck(Card, Deck): boolean \\
    + createCardToDeckForCategory(Category, Deck): boolean \\
    + getDeckByUUID(String): Deck \\
    + updateDeckData(Deck, Deck): boolean \\
    + updateStudySystem(Deck, StudySystem): boolean \\
    + addStudySystemTypeAndUpdate(StudySystemType): boolean \\
    + deleteDeck(Deck): boolean \\
    + deleteDecks(Deck[]): boolean \\
    + getDecksByCard(Card): Set<Deck> \\
    + getCardsInDeck(Deck): Set<Card> \\
    + numCardsInDeck(Deck): int \\
    + getStudySystemTypes(): Set<StudySystemType> \\
}{}
>>>>>>> 209d0be2ff9f5d6e3bebf4fe1b58322b02d32381

    \umlclass[x=25,y=-21.5]{CategoryLogic}
{
    + createCardToCategory(Card, Category): boolean \\
    + getCategories(): Set<Category> \\
    + getCategory(String): Category \\
    + getCategoryByUUID(String): Category \\
    + getCategoriesByCard(Card): Set<Category> \\
    + getCardsByCategory(Category): Set<Card> \\
    + numCardsInCategory(Category): int \\
    + updateCategoryData(Category, Category): boolean \\
    + deleteCategory(Category): boolean \\
    + deleteCategories(Category[]): boolean \\
    + deleteCardToCategory(CardToCategory): boolean \\
}{}
\end{umlpackage}{Logic}

\begin{umlpackage}[fill=beaublue, x=3, y=0]{Persistence}
    \umlclass[x=0, y=-30.5]{CardRepository}
{ 
    + findCardsByCategory(Category): List<Card>\\
    + findCardsByTag(Tag): List<Card>\\
    + getCards(long,long): List<Card>\\
    + findCardsWith(String): List<Card>\\
    + findCardByName(String): Card\\
    + updateCard(Card,Card): void\\
    + saveCard(Card): void\\
    + deleteCard(Card): void\\
    + findNumberOfDecksToCard(Card): int\\
}{}
    \umlclass[x=12, y=-30.5]{DeckRepository}
{ 
    + updateDeck(Deck,Deck): void\\
    + updateStudySystem(Deck,StudySystem): void\\
    + saveDeck(Deck): void\\
    + deleteDeck(Deck): void\\
    + getDecks(): List<Deck>\\
    + getDeck(int): Deck\\
}{}
    \umlclass[x=24,y=-22]{CategoryRepository}
{
    + createCategory(String, List<Category>, List<Category>): void \\
    + updateCategory(Category, Category): void \\
    + saveCategory(Category): void \\
    + deleteCategory(Category): void \\
    + getCategories(): List<Category> \\
    + getCategory(int): Category
}{}
\end{umlpackage}{Persistence}


\begin{umlpackage}[x=40, y=-30, fill=beaublue]{Datamodel}
\end{umlpackage}{Datamodel}

\umlassoc[attr1=1|\ , attr2=\ |1]{GUI}{Controller}
\umluniassoc[attr1=1|\ , attr2=\ |1]{Controller}{Logic}
\umluniassoc[attr1=1|\ , attr2=\ |1]{Logic}{Persistence} 
\umldep[stereo=use]{Persistence}{Datamodel}
\umldep[stereo=use, geometry=-|]{Logic}{Datamodel}
\umldep[stereo=use, geometry=-|]{Controller}{Datamodel}

\draw (15.2,-16) node {\Huge x};
\draw (15.15,-25.7) node {\Huge x};



%\umlemptyclass[x=6, y=-18]{DM:Card}
    \end{tikzpicture}
    \newpage


    \newcommand\user{Susi}
    \newcommand\editpage{e}
    \newcommand\testpage{t}
    \newcommand\deckpage{d}
    \newcommand\ssystem{s}
    \newcommand\control{c}
    \newcommand\logic{l}
    \newcommand\pers{p}
    \thispagestyle{empty}
    \hspace*{15cm}
    \begin{minipage}{50cm}
        \section*{Sequenzdiagramm 1: Karte erstellen}
        \vspace*{4cm}
        \begin{tikzpicture}
    \begin{umlseqdiag}
        \umlactor[class=User]{\user}
        \umlobject[class=EditCardPage]{\editpage}
        \umlobject[class=Controller]{\control}
        \umlobject[class=Logic]{\logic}
        \umlobject[class=Persistence]{\pers}
        %\umlmulti[class=C]{c}
        \begin{umlcall}[op=Will Karte erstellen, type=synchron]{\user}{\editpage}
            %
            % Get card data
            %
            \begin{umlcall}[op=getCardByUUID(), type=synchron, return=Card]{\editpage}{\control}
                \begin{umlcall}[op=getCardByUUID(), type=synchron, return=Card]{\control}{\logic}
                    \begin{umlfragment}[type=if, label=!empty, inner xsep=8, fill=green!10]
                        %\umlfpart[default]
                        \begin{umlcall}[op=getCardByUUID(), type=synchron, return=Card]{\logic}{\pers}
                        \end{umlcall}
                    \end{umlfragment}
                \end{umlcall}
            \end{umlcall}

            \begin{umlfragment}[type=if, name=showdelete, fill=green!10]
                \begin{umlcallself}[op=Löschen Button anzeigen, type=synchron]{\editpage}
                \end{umlcallself}
            \end{umlfragment}
            \umlnote[x=-3, y=-6] {showdelete}{getCardByUUID() $\ne$ null}

            
            %
            % get available categories
            %
            \begin{umlcall}[op=getCategories(), type=synchron, return=Set<Category>]{\editpage}{\control}
                \begin{umlcall}[op=getCategories(), type=synchron, return=Set<Category>]{\control}{\logic}
                        \begin{umlcall}[op=getCategories(), type=synchron, return=Set<Category>]{\logic}{\pers}
                        \end{umlcall}
                \end{umlcall}
            \end{umlcall}

            
            %
            % get available tags
            %
            \begin{umlcall}[op=getTags(), type=synchron, return=Set<Tag>]{\editpage}{\control}
                \begin{umlcall}[op=getTags(), type=synchron, return=Set<Tag>]{\control}{\logic}
                        \begin{umlcall}[op=getTags(), type=synchron, return=Set<Tag>]{\logic}{\pers}
                        \end{umlcall}
                \end{umlcall}
            \end{umlcall}
        \end{umlcall}

        %
        % Cancel
        %
        \begin{umlcall}[op=Abbrechen, type=synchron, return=Zurück zur übersicht]{\user}{\editpage}
        \end{umlcall}

        %
        % Apply changes
        %
        \begin{umlcall}[op=Übernehmen, type=synchron, return=Zurück zur übersicht]{\user}{\editpage}
            \begin{umlcall}[op=updateCardData(), type=synchron, return=boolean]{\editpage}{\control}
                \begin{umlcall}[op=updateCardData(), type=synchron, return=boolean]{\control}{\logic}
                    \begin{umlfragment}[type=if, label=uuid.isEmpty, inner xsep=10, fill=green!10]
                        \begin{umlcall}[op=saveCard(), type=synchron, return=boolean]{\logic}{\pers}
                        \end{umlcall}
                        \umlfpart[else]
                        \begin{umlcall}[op=updateCard(), type=synchron, return=boolean]{\logic}{\pers}
                        \end{umlcall}
                    \end{umlfragment}
                \end{umlcall}
            \end{umlcall}
        \end{umlcall}


        %\umlcreatecall[class=E, x=8]{\user}{e}
    \end{umlseqdiag}
\end{tikzpicture}
    \end{minipage}
    \newpage
    \thispagestyle{empty}
    %\hspace*{1.5\textwidth}
    \hspace*{5cm}\vspace*{2cm}
    \begin{minipage}{50cm}
        \section*{Sequenzdiagramm 2: Lernsystem lernen}
        \begin{tikzpicture}
    \begin{umlseqdiag}
        \umlactor[class=User]{\user}
        \umlobject[class=ViewSingleDeckPage]{\deckpage}
        \umlobject[class=TestCardPage]{\testpage}
        \umlobject[class=Controller]{\control}
        \umlobject[class=Logic]{\logic}
        \umlobject[class=Persistence]{\pers}
        
        \begin{umlcall}[op=Will Deck testen, type=synchron]{\user}{\deckpage}
            %
            % Test card
            %
            \begin{umlcall}[op=startTests(), type=synchron]{\deckpage}{\testpage}
                \begin{umlcall}[op=getCardByUUID(), type=synchron, return=Card]{\control}{\logic}
                    \begin{umlfragment}[type=alt, label=!empty, inner xsep=8, fill=green!10]
                        %\umlfpart[default]

                        \begin{umlcall}[op=getCardByUUID(), type=synchron, return=Card]{\logic}{\pers}
                        \end{umlcall}
                    \end{umlfragment}
                \end{umlcall}
            \end{umlcall}
        \end{umlcall}

        %\umlcreatecall[class=E, x=8]{\user}{e}
    \end{umlseqdiag}
\end{tikzpicture}
    \end{minipage}
    \recalctypearea
\end{document}

%\begin{umlfragment}
%    \begin{umlcall}[op=opb(), type=synchron, return=1]{b}{c}
%        \begin{umlfragment}[type=alt, label=condition, inner xsep=8, fill=green!10]
%            \begin{umlcall}[op=opc(), type=asynchron, fill=red!10]{c}{d}
%            \end{umlcall}
%            \begin{umlcall}[type=return]{c}{b}
%            \end{umlcall}
%            \umlfpart[default]
%            \begin{umlcall}[op=opd(), type=synchron, return=3]{c}{d}
%            \end{umlcall}
%        \end{umlfragment}
%    \end{umlcall}
%\end{umlfragment}
%\begin{umlfragment}
%    \begin{umlcallself}[op=ope(), type=synchron, return=4]{b}
%        \begin{umlfragment}[type=assert]
%            \begin{umlcall}[op=opf(), type=synchron, return=5]{b}{c}
%            \end{umlcall}
%        \end{umlfragment}
%    \end{umlcallself}
%\end{umlfragment}

%\begin{umlfragment}
%    \begin{umlcall}[op=opg(), name=test, type=synchron, return=6, dt=7, fill=red!10]{\user}{e}
%        \umlcreatecall[class=F, stereo=boundary, x=12]{e}{f}
%    \end{umlcall}
%    \begin{umlcall}[op=oph(), type=synchron, return=7]{\user}{e}
%    \end{umlcall}
%\end{umlfragment}