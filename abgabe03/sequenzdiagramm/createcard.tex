\begin{tikzpicture}
    \begin{umlseqdiag}
        \umlactor[class=User]{\user}
        \umlobject[class=EditCardPage]{\editpage}
        \umlobject[class=Controller]{\control}
        \umlobject[class=Logic]{\logic}
        \umlobject[class=Persistence]{\pers}
        %\umlmulti[class=C]{c}
        \begin{umlcall}[op=Will Karte erstellen, type=synchron]{\user}{\editpage}
            %
            % Get card data
            %
            \begin{umlcall}[op=getCardByUUID(), type=synchron, return=Card]{\editpage}{\control}
                \begin{umlcall}[op=getCardByUUID(), type=synchron, return=Card]{\control}{\logic}
                    \begin{umlfragment}[type=if, label=!empty, inner xsep=8, fill=green!10]
                        %\umlfpart[default]
                        \begin{umlcall}[op=getCardByUUID(), type=synchron, return=Card]{\logic}{\pers}
                        \end{umlcall}
                    \end{umlfragment}
                \end{umlcall}
            \end{umlcall}

            \begin{umlfragment}[type=if, name=showdelete, fill=green!10]
                \begin{umlcallself}[op=Löschen Button anzeigen, type=synchron]{\editpage}
                \end{umlcallself}
            \end{umlfragment}
            \umlnote[x=-3, y=-6] {showdelete}{getCardByUUID() $\ne$ null}

            
            %
            % get available categories
            %
            \begin{umlcall}[op=getCategories(), type=synchron, return=Set<Category>]{\editpage}{\control}
                \begin{umlcall}[op=getCategories(), type=synchron, return=Set<Category>]{\control}{\logic}
                        \begin{umlcall}[op=getCategories(), type=synchron, return=Set<Category>]{\logic}{\pers}
                        \end{umlcall}
                \end{umlcall}
            \end{umlcall}

            
            %
            % get available tags
            %
            \begin{umlcall}[op=getTags(), type=synchron, return=Set<Tag>]{\editpage}{\control}
                \begin{umlcall}[op=getTags(), type=synchron, return=Set<Tag>]{\control}{\logic}
                        \begin{umlcall}[op=getTags(), type=synchron, return=Set<Tag>]{\logic}{\pers}
                        \end{umlcall}
                \end{umlcall}
            \end{umlcall}
        \end{umlcall}

        %
        % Cancel
        %
        \begin{umlcall}[op=Abbrechen, type=synchron, return=Zurück zur übersicht]{\user}{\editpage}
        \end{umlcall}

        %
        % Apply changes
        %
        \begin{umlcall}[op=Übernehmen, type=synchron, return=Zurück zur übersicht]{\user}{\editpage}
            \begin{umlcall}[op=updateCardData(), type=synchron, return=boolean]{\editpage}{\control}
                \begin{umlcall}[op=updateCardData(), type=synchron, return=boolean]{\control}{\logic}
                    \begin{umlfragment}[type=if, label=uuid.isEmpty, inner xsep=10, fill=green!10]
                        \begin{umlcall}[op=saveCard(), type=synchron, return=boolean]{\logic}{\pers}
                        \end{umlcall}
                        \umlfpart[else]
                        \begin{umlcall}[op=updateCard(), type=synchron, return=boolean]{\logic}{\pers}
                        \end{umlcall}
                    \end{umlfragment}
                \end{umlcall}
            \end{umlcall}
        \end{umlcall}


        %\umlcreatecall[class=E, x=8]{\user}{e}
    \end{umlseqdiag}
\end{tikzpicture}